\documentclass[12pt, a4paper, twoside]{book}
\usepackage[utf8]{inputenc}
\usepackage{enumitem}
\usepackage{amsmath}
\newlist{steps}{enumerate}{2}
\setlist[steps]{label*=\arabic*.}

\title{EE543 - Neurocomputers and Deep Learning}
\author{Ahmet Taha Albayrak}
\date{December 2021}

\begin{document}

\begin{titlepage}
\maketitle
\end{titlepage}

\chapter{From Biological Neuron To Artificial Neuorn Model}
\chapter{Recurrent Neural Networks and Neurodynamis}
\chapter{Neural Networks as Associative Memory}

\chapter{Learning in Feed Forward Neural Networks}
\section{Perceptron Convergence Procedure}


Perceptron was introduced by Frank Rosenblatt in the late 1950's with a learning algorithm.
Perceptron may have continuous valued inputs.
It works in the the same way as the formal artificial neuron defined previously
Its activation is determined by equation:
\[\alpha= w^Tu+\theta \] 
Now, consider such a perceptron in N dimensional space,t he equation 
\[w^Tu + \theta = 0\] 
that 
\[w_1u_1 + w_2u_2 +...+w_nu_n=0\] defines a hyperplane. This hyperplane devides the input space into two parts
such that at one side, the perceptron has output value +1, and in the other side, it is -1.

A perceptron can be used to decide whether an input vector belongs to one of two classes say classes A and B.
The decision rule may be set as to respond as class A if the output is +1 and as class B if the output is -1.The perceptron forms two decision regions seperated by the hyperplane. The equation of the boundary hyperplane depends on the connection weights and threshold 

Connection weights and the threshold in a perceptron can be fixed or adapted using a number of different algorithms.
The original convergence procedure developed by Rosenblatt for adjusting weights is provided in the following
\pagebreak

\begin{steps}
    \item Initialize weight and threshold
    \item Present New Input and Desired output
    \item Calculate actual output
    \item Adapt weights
    \item Repeat steps 2-4 no error occurs
\end{steps}

In [Rosenblatt, 1959] it is proved that if the inputs presented from the two classes
are linearly separable, that is they fall on opposite side of some hyperplane, then the perceptron convergence
procedure always convergence in time. Furthermore, it positions the final decision hyperplane such that it separates the sample of class A
from those of class B \\

One problem with the perceptron convergence procedure is that decision boundary may oscillate continuously when the distribution overlap or the classes are not linearly separable.

\section{LMS Learning Rule}
A modification to the perceptron convergence procedure forms the Least Mean Square (LMS) solution for the case that the classes are not separable.
This solution minimizes the mean square error between the desired output and the acutal output of the processing element
The LMS algorithm was first proposed for Adaline (Adative Linear Element) in [Widrow and Hoff 60]. \\


The output function of the Adaline can be represented by the identity function as $f(a)=a$
So the output can be written in terms of input and weights as 
\[x=f(a)=\sum_{j = 0}^{N} w_ju_j\] 
where the bias is implemented via a connection to a constant input $u_0$ 
which means the input vector and the weight vector are of space $R^{(N+1)}$ instead of $R^N$ 
The output equation of Adaline can be written as: $ x=w^Tu$ where w and u are weight and input vectors respectively having dimension $N+1$ \\
  

Suppose that we have a set of input vectors $u^k, k=1,2..K$,each having its own desired output value $y^k$
The performance of the Adaline for a given input value $u^k$ can be defined by considering the difference between the desired output $y^k$ and 
the actual output $x^k$, which is called error and denoted as $\varepsilon^k$
Therefore, the error for the input $u^k$ is as follows: 
\[\varepsilon^k=y^k-x^k=y^k-w^Tu^k\]

The aim of LMS learning is to adjust the weights through a training set {{$u^k$,$y^k$}} $k=1,2..K$ such that the mean of square of the errors is minimum.
The mean square error is defined as 
\[<(\varepsilon^k)^2>= \lim_{k \to \infty} 1/K \sum_{n = 1}^{K}(\varepsilon^k)^2\] where the notation $<.>$ denotes the mean value. 

The mean square error can be written as $<(\varepsilon^k)^2> = <(y^k - w^Tu^k)^2> = <(y^k)^2> + w^T <u^k\times u^k> w - 2 <y^ku^(k^T)> w$ where T denotes transpose and $\times$ denotes outer vector product.
Defining input correlation matrix $R$ and a vector $P$ as $R = <u^k\times u^k> = <u^ku^(k^T)>$ and $P = <y^ku^k>$ results in $e(w)=<(\varepsilon^k)^2> = <(y^k)^2> + w^TRw - 2P^Tw$. 
The optimum value $w^*$ for weight vector corresponding to the minimum of the mean square error can be obtained by evaluating the gradient of $e(w)$. \\

The point which makes gradient zeros gives us the value of $w^*$. That is 
\[\nabla e(w)|_{w=w^*} = \frac{d e(w)}{d w}|_{w=w^*} = 2Rw^* - 2P = 0 \]. Here the gradient is $\nabla e(w) = [ \frac{d e}{ d w_1} \frac{d e}{ d w_2} ... \frac{d e}{ d w_n} ]^T$ and it is a vector exteding in the direction of the greatest rate of change
The gradient of a function evaluated at some point is zero if the function has a maximum or minimum at that point. 
The error function is of the second degree. So it is a paraboloid and it has a single minimum at point $w^*$.

\section{Stepest Decent Algorithm}
The analytical calculation of the optimum weight vector for a problem is rather difficult in general. Not only does the matrix manipulation get cumbersome for the large dimensions, but also
each component of R and P require knowledge of the statistics of the input signal [Freeman 91]. A better approach would be to let the Adaline Linear Combiner to find the optimum weights by itself a search over the error surface. \\

Instead of having a purely random search, some intelligence is added to procedure such that weight vector is changed by considering the gradient of $e(w)$ iteratively [Widrow 60], according to formula know as \textbf{delta rule}. $w(t+1)=w(t)+\Delta w(t)$ where $\Delta w(t) = - \eta\nabla e(w(t)) $. In the formula $\eta$ is a small positive constant \\

For the real valued scalar function $e(w)$ on a vector space $w \in R^N$, the gradient $e(w)$ gives the direction of the steepest upward slope, so the negative of the gradient is the direction of the steepest descent. 
We have considered the linear output function in the derivation of the optimum weight $w^*$ for the minimum error. However in the general case, we should consider any nonlinearity $f(.)$ at the output of the neuron.
It should be noted that in such a case the error surface is no more paraboloid, so it may have several local minima.
For an input $u^k$ applied at the time $t$, $(e^k(t))^2$ can be used as an approximation to $<\varepsilon(k)^2>$ where $\varepsilon^k(t))=y^k - f(a^k) = y^k - f(w(t)^Tu^k)$. Therefore, we obtain, $\nabla <(\varepsilon^k)^2 > \simeq \nabla(\varepsilon^k(t))^2 = \nabla(y^k - f(a^k)^2$ with differtiable function $f(.)$ having derivate $f'(.)$, it becomes $\nabla(y^k-f(a))^2= -2\varepsilon^k(t)f'(a)\nabla a$ \\

Since $\nabla a^k = \nabla w(t)^Tu^k=u^k$ the weight update formula becomes $ w(t+1) = w(t) + 2\eta\varepsilon^k(t)f'(a)u^k$

\begin{steps}
    \item Apply an input vector $u^k$ with the desired output value $y^k$ to the neuron's input.
    \item By considering $u^k$ and using the current value of the weight vector determine the value of the activation vector $a^k$ where $a^k=w(t)^Tu^k$
    \item Determine the value of the derivation of the output function using the current value of activation $a^k$ that is $f'(a^k)= \frac{f(a)}{a}|_{a=a^k}$.
    \item Determine the value of error $\varepsilon^k(t)$ as $\varepsilon^k(t) = y^k - f(a^k)$
    \item Update the weight vector wtih respect to following update formula $w(t+1) = w(t) + 2\eta f'(a^k)\varepsilon^k(t)u^k$.
    \item Repeat steps 1-5 until $<\varepsilon^k(t)>$ reduces to an acceptable level.
\end{steps}

Notice that the iterative weight update by the \textbf{delta rule} is derived by assumingh constant $u^k$. Therefore, it tends to minimize the error wtih respect to applied $u^k$
In fact, we require the average error to be minimized. This implies that 
\[\frac{de}{dw_j}=  \frac{1}{K} \sum_{k = 1}^{K}\frac{d(\varepsilon^k)^2}{dw_j} = \dfrac{1}{K} \sum_{k=1}^{K} \frac{2\varepsilon^kd\varepsilon^k}{dw_j}\]

Therefore, the net change in $w_j$ after one complete cycle of pattern presentation is expected to be
\[w_j(t+K) = w_j(t) - \eta\frac{1}{K}\sum_{k=1}^{K}\frac{2\varepsilon^kd\varepsilon^k}{w_j}\]
However, this would be true that if the weights are not updated along a cycle but only at the end. By changing the weights as each pattern is presented, we depart to some extend from gradient descent in $e$
Nevertheless, provided the learning rate is sufficiently small, this departure will be negligible and \textbf{delta rule} will implement a very close approximation to gradient descent in mean squared error [Freeman 91].

\section{The Backpropagation Algorithm: Single Layer Network}

Consider a single layer multiple output network. Here, we stil have N inputs denoted $u_j$, $j=1...N$ but M processing elements whose activation and outputs denoted as $a_i$ and $x_i$, $i=1...M$ respectively
$w_{ji}$ used to denote the strength of the connection from the $j^{th}$ input to the $i^{th}$ processing element. 
In vector notation $w_{ji}$ is the $j^{th}$ component of weight vector $w_i$, while $u_j$ is the $j^{th}$ component of the input vector $u$ \\

Let $u^k$ and $y^k$ to represent the $k^{th}$ input sample and the corresponding desired output vector respectively.
Let the error observed at the output $i$, when $u^k$ is applied at the input, be $\varepsilon_i^k = y_i^k-x_i^k$. If the error is to be written in terms of the input vector $u^k$ and the weights $w_j$, we obtain $\varepsilon_i^k = y_i^k - f(w_i^Tu^k)$.

If we take the partial derivative with respect to $w_ji$ by applying the chain rule 
\[\frac{d\varepsilon_i^k}{dw_{ji}} = \frac{d\varepsilon_i^k}{dx_i^k}\frac{dx_i^k}{dw_{ji}}\]
where 
\[\frac{d\varepsilon_i^k}{dx_i^k} = -1 \ and\ \frac{dx_i^k}{dw_{ji}} = f'(a_i^k)u^k_j \] 
We obtain
\[\frac{d\varepsilon_i^k}{dw_{ji}} = -f'(a^k)u_j^k\]

If we define the total output error for input $u^k$ as the sum of the square of errors at each neuron output, then partial
derivate of the total error with respect to $w_{ji}$, when $u^k$ is applied at the input can be written as
\[\frac{de^k}{dw_{ji}} = \frac{de^k}{d\varepsilon_i^k}\frac{d\varepsilon_i^k}{dw_{ji}}\]
which is 
\[\frac{de^k}{dw_{ji}} = -\varepsilon_i^kf'(a^k)u_j\]
by defining 
\[\delta_i^k = \varepsilon_i^kf'(a^k)\]
it can be formulated as 
\[\frac{de^k}{dw_{ji}}= -\delta_i^ku_i^k\]

For the error to be minimum the gradient of the total error with respect to weights should be 0.
In order to reach the minimum of the total error, without solving the above equation, we can apply \textbf{delta rule} in the same way explained for the steepest descent algorithm
\[w_{ji}(t+1) =  w_{ji}(t) + \eta\delta_i^ku_j^k \ for j=1..N, i=1..M\]

\section{The Backpropagation Algorithm: Multi Layer Network}

Now, assume that another layer of neurons is connected to the input side of the output layer.
Therefore we have input, hidden and the output layer. In order to discriminate between the elements of the hidden and output layers
We will use the subscripts $L$ and $o$ respectively.
Furthermore, we will use h as the index on the hidden layer elements, while still using index $j$ and $i$ for the input and output layers respectively

In such a network, the output value of $i^{th}$ neuron of the output layer can be written as 
\[x_{i_o}^k=f_o(w_{i_o}^Tx_L^k)\]
where $x_L^k$ being the vector of the output values at hidden layer that is connected as input to the output layer. The value of the $h^{th}$ element $x_L^k$ is determined by the equation
\[x_{h_L}^k=f_L(w_{h_L}^Tu^k)\]
The partial derivate of the output of a neuron $i_o$ of output layer wtih respect the hidden layer weight $w_{jhL}$ can be determined by applying chain rule
\[\frac{dx_{i_o}^k}{dw_{jh_L}} = \frac{dx_{i_o}^k}{dx_{h_L}^k}  \frac{dx_{h_L}^k}{dw_{jh_L}} \]
We can rewrite the equation as 
\[\frac{dx_{i_o}^k}{dw_{jh_L}} = (f'_o(a_{i_o}^k)w_{h_Li_o})(f_L'(a_{h_L}^ku_j^k))\]

Then the partial derivate of the total error with respect to hidden layer $w_{jh_L}$ can be written as
\[\frac{de^k}{dw_{jh_L}}= - \sum_{i_0=1}^{M}\varepsilon_{i_o}^kf_o'(a_{i_o}^k)w_{h_Li_o}f_L'(a_{h_L}^k)u_j^k\] 

Therefore, the weight update rule for the hidden layer
\[w_{jh_L}(t+1)= w_{jh_L}(t) - \eta \frac{de^k}{dw_{jh_L}}\]
can be reformulated by using the same analogy in the single layer network section
\[w_{jh_L}(t+1) = w_{jh_L}(t) + \eta\delta_{h_L}^ku_j\]


\begin{steps}
    \item  \textbf{Initialize weights:} to small random values
    \item  \textbf{Apply a sample:} apply to the input a sample vector $u^k$ having desired output vector $y^k$
    \item  \textbf{Forward Phase:} Starting from the first hidden layer and propagating towards the output layer
    \begin{steps}
        \item \textbf{Calculate the activation values} for the units at layer $L$
        \item \textbf{Calculate the output values} for the units at layer $L$
    \end{steps} 
    \item \textbf{Output erros:} Calculate the error at the output layer as
    \[\delta_{i_o}^k = (y_{i_o}^k - x_{i_o}^k)f_o'(a_{i_o}^k)\]
    \item \textbf{Backward Phase} Propagate error backward to the input layer through each layer $L$ using the error term
    \[\delta_{h_L}^k = f_L'(a_{h_L}^k)\sum_{i_{L+1}=1}^{N_{L+1}}\delta_{i_{(L+1)}}^k w_{h_{L_{(L+1)}^i}}^k\]
    \item \textbf{Weight update:} Update the weights according to the formula
    \[w_{j_{(L-1)h_L}}(t+1) = w_{j_{(L-1)h_L}}(t) +\eta\delta_{h_L}^kx_{j_{(L-1)}}^k \] 
    \item \textbf{Repeat} steps 1-6 until the stop criterion is satisfied.
\end{steps}



\end{document}